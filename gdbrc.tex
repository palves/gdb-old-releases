%This file is TeX source for a reference card describing GDB, the GNU debugger.
%Copyright (C) 1991 Free Software Foundation, Inc.
%Permission is granted to make and distribute verbatim copies of
%this reference provided the copyright notices and permission notices
%are preserved on all copies.
%
%TeX markup is a programming language; accordingly this file is source
%for a program to generate a reference.
%
%This program is free software; you can redistribute it and/or modify
%it under the terms of the GNU General Public License as published by
%the Free Software Foundation; either version 1, or (at your option)
%any later version.
%
%This program is distributed in the hope that it will be useful, but
%WITHOUT ANY WARRANTY; without even the implied warranty of
%MERCHANTABILITY or FITNESS FOR A PARTICULAR PURPOSE.  See the GNU
%General Public License for more details.
%
%You can find a copy of the GNU General Public License in the GDB
%manual; or write to the Free Software Foundation, Inc.,
%675 Mass Ave, Cambridge, MA 02139, USA.
%
%You can contact the author as:  pesch@cygnus.com
%
%                                Roland Pesch
%                                Cygnus Support
%                                814 University Ave.
%                                Palo Alto, CA 94301 USA
%
%                                +1 415 322 3811
%
%            Cygnus Support is an organization devoted to commercial
%            support of free software.  For general information
%            contact ``info@cygnus.com''
%
%
\input threecol
{%
\def\$#1${{#1}}%   Kluge: collect RCS revision info without $...$
\xdef\manvers{\$Revision: 1.1 $}%
}
\vsize=8in
\hyphenpenalty=5000\tolerance=2000\raggedright
%
\font\bbf=cmbx10
\font\vbbf=cmbx12
\font\smrm=cmr6
\font\brm=cmr10
\font\rm=cmr8
\font\it=cmti8
\font\tt=cmtt8
\normalbaselineskip=9pt\baselineskip=9pt
%
\parindent=0pt
\parskip=0pt
\footline={\vbox to0pt{\hss}}
%
\def\ctl#1{{\tt C-#1}}
\def\opt#1{{\brm[{\rm #1}]}}
\def\xtra#1{\noalign{\smallskip{\tt#1}}}
%
\long\def\sec#1;#2\endsec{\vskip 1pc
\halign{%
%COL 1 (of halign):
\vtop{\hsize=1in\tt
##\par\vskip 2pt
}\quad
%COL 2 (of halign):
&\vtop{\hsize=2.1in\hangafter=1\hangindent=0.5em
\rm ##\par\vskip 2pt}\cr
%Tail of \long\def fills in halign body with \sec args:
\noalign{{\bbf #1}\vskip 2pt}
#2
}
}

{\vbbf GDB QUICK REFERENCE}
\vskip 5pt
{\smrm GDB Version 4.00---Cygnus Support 1991}

\sec Starting GDB;
gdb&starts GDB, with no debugging files\cr
gdb {\it program}&begin debugging {\it program}\cr
gdb {\it program core}&debug coredump {\it core} produced by {\it program}\cr
\endsec

\sec Stopping GDB;
quit&Exit GDB; abbreviate as {\tt q} or {\tt EOF}\par (eg \ctl{d})\cr
INTERRUPT&(eg \ctl{c}) terminate current command\cr
\endsec

\sec Getting Help;
help&List classes of commands\cr
help {\it class}&One-line descriptions for commands in {\it class}\cr
help {\it command}&Describe {\it command}\cr
\endsec

\sec Executing your Program;
run {\it arglist}&start your program with {\it arglist}\cr
run&start your program with current argument list\cr
run $\ldots$ <{\it inf} >{\it outf}&start program with input, output
redirected\cr
\cr
kill&Kill running program\cr
\cr
tty {\it dev}&Use {\it dev} as default i/o for next {\tt run}\cr
set args {\it arglist}&specify {\it arglist} for next
{\tt run}\cr
set args&specify empty argument list\cr
show args&display argument list\cr
\cr
show environment&show all environment variables\cr
show env {\it var}&show value of environment variable {\it var}\cr
set env {\it var} {\it expr}&set environment variable {\it var}\cr
delete env {\it var}&Remove {\it var} from environment\cr
\endsec

\sec Shell Commands;
cd {\it dir}&Change working directory to {\it dir}\cr
pwd&Print working directory\cr
make $\ldots$&Call ``{\tt make}''\cr
shell {\it cmd}&Execute arbitrary shell command string\cr
\endsec

\vfill
\centerline{\smrm \copyright 1991 Free Software Foundation, Inc.\qquad Permissions on back}
\eject
\sec Breakpoints and Watchpoints;
break \opt{\it file\tt:}{\it line}&Set breakpoint at {\it line} number \opt{in \it file}\par 
eg:\quad{\tt break main.c:37}\quad\cr
break \opt{\it file\tt:}{\it fun}&Set breakpoint at {\it
fun}() \opt{in \it file}\cr
break +{\it offset}\par
break -{\it offset}&Set break at offset from current stop\cr
break *{\it addr}&Set breakpoint at address {\it addr}\cr
break&Set breakpoint at next instruction\cr
break $\ldots$ if {\it expr}&Break conditionally on nonzero {\it expr}\cr
cond {\it bno} \opt{\it expr}&New conditional expression on breakpoint
number {\it bno}; make unconditional if no {\it expr}\cr
tbreak $\ldots$&Temporary break; disable when reached\cr
rbreak {\it regex}&Break on all functions matching {\it regex}\cr
watch {\it expr}&Set a watchpoint for expression {\it expr}\cr
catch {\it x}&Set breakpoint at C++ handler for exception {\it x}\cr
\cr
info break&Show defined breakpoints\cr
info watch&Show defined watchpoints\cr
\cr
clear&Delete breakpoints at next instruction\cr
clear \opt{\it file\tt:}{\it fun}&Delete breakpoints at entry to {\it fun}()\cr
clear \opt{\it file\tt:}{\it line}&Delete breakpoints on source line \cr
delete \opt{{\it bnos}}&Delete breakpoints numbered {\it bnos};
\opt{or all breakpoints}\cr
\cr
disable \opt{{\it bnos}}&Disable breakpoints {\it bnos} \opt{or all}\cr
enable {\it bnos}&Enable breakpoints {\it bnos} \opt{or all}\cr
enable once {\it bnos}&Enable breakpoints; disable again when
reached\cr
enable del {\it bnos}&Enable breakpoints; delete when reached\cr
\cr
ignore {\it bno} {\it count}&Ignore breakpoint number {\it bno}, {\it count}
times\cr
\cr
commands {\it bno}\par
\qquad {\it command list}&Execute GDB {\it command list} every time breakpoint {\it bno} is reached\cr
end&(use only with {\tt commands}) End of {\it command list}\cr
\endsec

\sec Signals;
handle {\it signal} {\it act}&Specify GDB actions when {\it signal} occurs:\cr
\quad print&Announce when signal occurs\cr
\quad noprint&Be silent when signal occurs\cr
\quad stop&Halt execution on signal\cr
\quad nostop&Do not halt execution\cr
\quad pass&Allow your program to handle signal\cr
\quad nopass&Do not allow your program to see signal\cr
info signal&Show table of signals and GDB action for each\cr
\endsec

\vfill\eject
\sec Execution Control;
continue \opt{\it count}\par
c \opt{\it count}&Continue running; if {\it count} specified, ignore
this breakpoint next {\it count} times\cr
\cr
step \opt{\it count}\par
s \opt{\it count}&Execute until another line reached; repeat {\it count} times if
specified\cr
\cr
stepi \opt{\it count}\par
si \opt{\it count}&Step by machine instructions rather than source
lines\cr
\cr
next \opt{\it count}\par
n \opt{\it count}&Execute next line, including any function calls.\cr
\cr
nexti \opt{\it count}\par
ni \opt{\it count}&Next machine instruction rather than source
line\cr
\cr
until \opt{\it location}&Run until next instruction (or {\it
location}) reached\cr
\cr
finish&Run until selected stack frame returns\cr
return \opt{\it expr}&Pop selected stack frame without executing,
optionally setting return value\cr
\cr
signal {\it num}&Resume execution with signal {\it num} (none if {\tt 0})\cr
jump {\it line}\par
jump *{\it address}&Resume execution at specified {\it line} number or
{\it address}\cr
set var {\it expr}&Evaluate {\it expr} without displaying it; use for
altering program variables\cr
\endsec

\sec Debugging Targets;
target {\it type} {\it param}&Connect to target machine, process, or file\cr
info targets&Display available targets\cr
attach {\it param}&Connect to another target of same type\cr
detach&Release target from GDB control\cr
\endsec

\sec Expressions;
{\it expr}&An expression in C or C++ (including function calls), or:\cr
{\it addr\/}@{\it len}&An array of {\it len} elements beginning at {\it
addr}\cr
{\it file}::{\it nm}&A variable or function {\it nm} defined in {\it
file}\cr
$\tt\{${\it type}$\tt\}${\it addr}&Read memory at {\it addr} as specified
{\it type}\cr
print \opt{\tt/{\it f}\/} {\it expr}\par
p \opt{\tt/{\it f}\/} {\it expr}&Display the value of an expression\par 
in format {\it f}:\cr
\qquad x&hexadecimal\cr
\qquad d&signed decimal\cr
\qquad u&unsigned decimal\cr
\qquad o&octal\cr
\qquad a&address, absolute and relative\cr
\qquad c&character constant\cr
\qquad f&floating point\cr
call \opt{\tt /{\it f}\/} {\it expr}&Like {\tt print} but does not display
{\tt void}\cr
\endsec

\vfill\eject
\sec Memory;
x \opt{\tt/{\it Nuf}\/} {\it expr}&Examine memory at address {\it expr};
optional format spec follows slash.\cr
\quad {\it N}&Count of how many units to display;\cr
\quad {\it u}&Unit size; one of\cr
&{\tt\qquad b}\ individual bytes\cr
&{\tt\qquad h}\ halfwords (two bytes)\cr
&{\tt\qquad w}\ words (four bytes)\cr
&{\tt\qquad g}\ giant words (eight bytes)\cr
\quad {\it f}&Printing format.  Any {\tt print} format, or\cr
&{\tt\qquad s}\ Null-terminated string\cr
&{\tt\qquad i}\ Machine instructions\cr
disassem \opt{\it addr}&Display range of memory as machine
instructions; function surrounding {\it addr} or program counter, or range between two arguments\cr
\endsec

\sec Automatic Display;
display \opt{\tt/\it f\/} {\it expr}&Show value of {\it expr} each time
program stops \opt{according to format {\it f}\/}\cr
display&Display all enabled expressions on list\cr
undisplay {\it dnos}&Remove number(s) {\it dnos} from list of
automatically displayed expressions\cr
disable dis {\it dnos}&Disable display for expression(s) number {\it
dnos}\cr
enable dis {\it dnos}&Enable display for expression(s) number {\it
dnos}\cr
info display&Show numbered list of expressions to display\cr
\endsec

\sec Program Stack;
backtrace \opt{\it n}\par
bt \opt{\it n}&Print trace of all frames in stack; or of {\it n}
frames---innermost if {\it n}{\tt >0}, outermost if {\it n}{\tt <0}\cr
frame \opt{\it n}&Select frame number {\it n} or frame at address {\it
n}; if no {\it n}, display current frame\cr
up {\it n}&Select frame {\it n} frames up\cr
down {\it n}&Select frame {\it n} frames down\cr
info frame \opt{\it addr}&Description of selected frame, or frame at
{\it addr}\cr
info args&Arguments of selected frame\cr
info locals&Local variables of selected frame\cr
info catch&Exception handlers active in selected frame\cr
\endsec

\sec Symbol Table;
info address {\it s}&Show where symbol {\it s} is stored\cr
info func \opt{\it regex}&Show names, types of defined functions
(all, or matching {\it regex})\cr
info var \opt{\it regex}&Show names, types of global variables (all,
or matching {\it regex})\cr
info sources&Show all sources having debugging information\cr
whatis {\it expr}\par
ptype {\it expr}&Show data type of {\it expr} without evaluating; {\tt
ptype} gives more detail\cr
ptype {\it type}&Describe type, struct, union, or enum\cr
\endsec

\vfill\eject
\sec Controlling GDB;
set {\it param} {\it expr}&Set one of GDB's internal parameters,
controlling its interaction with you\cr
show {\it param}&Display current setting of a GDB parameter\cr
\xtra{\rm Parameters understood by {\tt set} and {\tt show}:}
\quad addressp {\it on/off}&print memory addresses in stacks,
structs\cr
\quad array-max {\it limit}&Number of elements to display from an
array\cr
\quad arraypr {\it off/on}&Compact or attractive format for
arrays\cr
\quad caution {\it on/off}&Enable or disable cautionary queries\cr
\quad editing {\it on/off}&Control {\tt readline} command-line editing\cr
\quad history&({\tt h}) covers a number of options:\cr
\quad h exp {\it off/on}&Disable or enable {\tt readline} history expansion\cr
\quad h file {\it filename}&File for recording GDB command history\cr
\quad h size {\it size}&Number of commands kept in history list\cr
\quad h write {\it off/on}&Control use of external file for
command history\cr
\cr
\quad pretty {\it off/on}&Compact or indented format for struct
display\cr
\quad prompt {\it str}&Use {\it str} as GDB prompt\cr
\quad radix {\it base}&Octal, decimal, or hex number representation\cr
\quad screen-h {\it lpp}&Number of lines before pause in
display\cr
\quad screen-w {\it cpl}&Number of characters before line folded\cr
\quad unionpr {\it on/off}&Enable or disable display of unions in
structs\cr
\quad verbose {\it on/off}&Control messages when loading
symbol table\cr
\quad vtblpr {\it off/on}&Display of C++ virtual function tables\cr
info editing&Show last 10 commands\cr
info editing {\it n}&Show 10 commands around number {\it n}\cr
info editing +&Show next 10 commands\cr
\endsec

\sec Working Files;
file {\it name}&Use {\it file} for symbols and executable\cr
core {\it name}&Read {\it file} as coredump\cr
exec {\it name}&Use {\it file} as executable only\cr
symbol {\it name}&Use only symbol table from {\it file}\cr
load {\it file} {\it addr}&Read additional symbols from {\it file},
dynamically loaded at {\it addr}\cr
info files&Display working files and targets in use\cr
\cr
share \opt{\it regex}&Add symbol information for shared libraries
matching {\it regex}, or all shared libraries\cr
info share&List names of shared libraries currently loaded\cr
\endsec

\vfill\eject
\sec Source Files;
dir {\it name}&Add directory {\it name} to front of source path\cr
dir&Clear source path\cr
info dir&Show current source path\cr
\cr
list&Show next ten lines of source\cr
list -&Show previous ten lines\cr
list {\it lines}&Display source centered around {\it lines}, 
specified as one of:\cr
\quad{\opt{\it file\tt:}\it num}&Line number \opt{in named file}\cr
\quad{\opt{\it file\tt:}\it function}&Beginning of function \opt{in
named file}\cr
\quad{\tt +\it off}&{\it off} lines after last printed\cr
\quad{\tt -\it off}&{\it off} lines previous to last printed\cr
\quad{\tt*\it address}&Line containing {\it address}\cr
list {\it f},{\it l}&from line {\it f} to line {\it l}\cr
info line {\it num}&Show starting, ending addresses of compiled code for
source line {\it num}\cr
forw {\it regex}&Search following source lines for {\it regex}\cr
rev {\it regex}&Search preceding source lines for {\it regex}\cr
\endsec

\sec GDB under GNU Emacs;
M-x gdb&Run GDB under Emacs\cr
\ctl{h} m&Describe GDB mode\cr
M-s&Step one line ({\tt step})\cr
M-n&Next line ({\tt next})\cr
M-i&Step one instruction ({\tt stepi})\cr
\ctl{c} \ctl{f}&Finish current stack frame ({\tt finish})\cr
M-c&Continue ({\tt cont})\cr
M-u&Up {\it arg} frames ({\tt up})\cr
M-d&Down {\it arg} frames ({\tt down})\cr
\ctl{x} SPC&(in source file) set break at point\cr
\endsec


\vfill
{\smrm\parskip=6pt
\centerline{Copyright \copyright 1991 Free Software Foundation, Inc.}
\centerline{Roland Pesch (pesch@cygnus.com), January 1991---\manvers}

This card may be freely distributed under the terms of the GNU
General Public License.

Please contribute to development of this card by annotating it.

No author assumes any responsibility for any errors on this card.}
\end
